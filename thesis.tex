%% ----------------------------------------
%%
%% NYU PhD thesis template.
%% Created by José Koiller 2007--2008.
%% Modified by Siddharth Krishna, 2019.
%%
%% ----------------------------------------


%% Use the first of the following lines during production to
%% easily spot "overfull boxes" in the output. Use the second
%% line for the final version.
%\documentclass[12pt,draft,letterpaper]{report}
\documentclass[12pt,oneside,letterpaper]{report}


% ----------------------------------------
% Macro to switch between draft version and final version
% ----------------------------------------

% Use or comment this to enable/disable draft version
% \def\draftversion{}
\newcommand{\draftfinal}[2]{\ifdefined\draftversion#1\else#2\fi}
\newcommand{\draftonly}[1]{\draftfinal{#1}{}}
\newcommand{\finalonly}[1]{\draftfinal{}{#1}}


% ----------------------------------------
% Thesis metadata
% ----------------------------------------

%% Replace the title, name, advisor name, graduation date and dedication below
%% with your own. Graduation months must be January, May or September.
\newcommand{\thesistitle}{On Complete Systems of Invariants for Ternary Biquadratic Forms}
\newcommand{\thesisauthor}{Amalie Emmy Noether}
\newcommand{\thesisadvisor}{Professor Paul Gordan}
\newcommand{\thesisdept}{Mathematics}
\newcommand{\gradmonth}{September}
\newcommand{\gradyear}{2019}
%% If you do not want a dedication, scroll down and comment out
%% the appropriate lines in this file.
\newcommand{\thesisdedication}{To my dog Weierstra\ss, with affection.}


% ----------------------------------------
% Layout and formatting
% ----------------------------------------

% Uncomment to get a big black box to spot "overfull hboxes"
% \setlength{\overfullrule}{5pt}


%% Page layout (customized to letter paper and NYU requirements):
\RequirePackage[margin=1in, includefoot, letterpaper]{geometry}


%% Color definitions:
\RequirePackage[prologue]{xcolor}
\definecolor[named]{ThesisBlue}{cmyk}{1,0.1,0,0.1}
\definecolor[named]{ThesisYellow}{cmyk}{0,0.16,1,0}
\definecolor[named]{ThesisOrange}{cmyk}{0,0.42,1,0.01}
\definecolor[named]{ThesisRed}{cmyk}{0,0.90,0.86,0}
\definecolor[named]{ThesisLightBlue}{cmyk}{0.49,0.01,0,0}
\definecolor[named]{ThesisGreen}{cmyk}{0.20,0,1,0.19}
\definecolor[named]{ThesisPurple}{cmyk}{0.55,1,0,0.15}
\definecolor[named]{ThesisDarkBlue}{cmyk}{1,0.58,0,0.21}

% School color found from university's graphic identity site:
% http://www.nyu.edu/employees/resources-and-services/media-and-communications/styleguide.html
\definecolor{SchoolColor}{rgb}{0.3412, 0.0235, 0.5490} % purple
\definecolor{chaptergrey}{rgb}{0.2600, 0.0200, 0.4600} % dialed back a little
\definecolor{midgrey}{rgb}{0.4, 0.4, 0.4}

\usepackage{hyperref}
\hypersetup{colorlinks,
  linkcolor=ThesisDarkBlue,
  citecolor=ThesisPurple,
  urlcolor=ThesisDarkBlue,
  filecolor=ThesisDarkBlue}


%% Captions of Figures, tables
\RequirePackage[labelfont={bf,sf,small,singlespacing},
                textfont={sf,small,singlespacing},
                % justification={justified,RaggedRight},
                % singlelinecheck=false,
                margin=0pt,
                figurewithin=chapter,
                tablewithin=chapter]{caption}

%% Chapter headings, captions
\usepackage{fix-cm}
\RequirePackage[raggedright,sc]{titlesec}
\definecolor{gray75}{gray}{0.75}
\newcommand{\hsp}{\hspace{20pt}}

\titleformat{\chapter}[hang]
{\Huge\sc}
{\textcolor{SchoolColor}{\thechapter}\hsp\textcolor{gray75}{|}\hsp}
{0pt}{\Huge\sc\raggedright}
% [\textcolor{gray75}{|}\hsp\textcolor{SchoolColor}{\thechapter}]


%% The following makes chapters and sections, but not subsections,
%% appear in the TOC (table of contents). Increase to 2 or 3 to
%% make subsections or subsubsections appear, respectively. It seems
%% to be usual to use the "1" setting, however.
\setcounter{tocdepth}{2}

%% Sectional units up to subsubsections are numbered. To number
%% subsections, but not subsubsections, decrease this counter to 2.
\setcounter{secnumdepth}{3}

%% Use the following commands, if desired, during production.
%% Comment them out for final version.
%\usepackage{layout} % defines the \layout command, see below
%\setlength{\hoffset}{-.75in} % creates a large right margin for notes and \showlabels

%% Controls spacing between lines (\doublespacing, \onehalfspacing, etc.):
\usepackage{setspace}

%% Use the line below for official NYU version, which requires
%% double line spacing. For all other uses, this is unnecessary,
%% so the line can be commented out.
\finalonly{
  \doublespacing % requires package setspace, invoked above
}

%% For generating sample text.
%% Can be removed when you've replaced all \lipsum commands with your text.
\usepackage{lipsum}


% ----------------------------------------
% Comments and TODOs:
% ----------------------------------------

% Uncomment this to remove all comments
\newcommand{\nocomments}{}

% Uncomment this to remove all TODOs
\newcommand{\notodos}{}

% Comments and TODOs
\newcommand{\fcomment}[2]{\ifdefined\nocomments{}\else\footnote{\textcolor{red}{#1:} #2}\fi}
\newcommand{\todo}[1]{\ifdefined\notodos{}\else\textcolor{red}{TODO\ifstrempty{#1}{}{: #1}}\fi}
\newcommand{\ftodo}[1]{\ifdefined\notodos{}\else\fcomment{TODO}{#1}\fi}

% Author comments:
\newcommand{\aen}[1]{\fcomment{Emmy}{#1}}


% ----------------------------------------
% User-specific packages and macros
% ----------------------------------------

%% This inputs your auxiliary file with \usepackage's and \newcommand's:
%% It is assumed that that file is called "defs.tex".
% ----------------------------------------
% Packages
% ----------------------------------------

% 
% Place here your \usepackage's. Some recommended packages are already included.
%

% Graphics:
\ifdefined\draftversion
\usepackage[draft]{graphicx}
\else
\usepackage{graphicx}
\fi
\usepackage{hyperref}
\hypersetup{
    colorlinks,
    citecolor=black,
    filecolor=black,
    linkcolor=black,
    urlcolor=black
}

%\usepackage{graphicx} % use this line instead of the above to suppress graphics in draft copies
%\usepackage{graphpap} % \defines the \graphpaper command

% Uncomment this to indent first line of each section:
% \usepackage{indentfirst}

% Good AMS stuff:
\usepackage{amsthm} % facilities for theorem-like environments
\usepackage[tbtags]{amsmath} % a lot of good stuff!

% Fonts and symbols:
\usepackage{amsfonts}
\usepackage{amssymb}

% Set the fonts
\RequirePackage[T1]{fontenc}
\ifxetex
  \RequirePackage[tt=false]{libertine}
\else
  \RequirePackage[tt=false, type1=true]{libertine}
\fi
\RequirePackage[varqu]{zi4}
\RequirePackage[libertine]{newtxmath}


% For typesetting inference rules
\usepackage{mathpartir}
% \usepackage{pftools}  % A local package
\newcommand{\bmmax}{2}
\usepackage{bm}

% Formatting tools:
%\usepackage{relsize} % relative font size selection, provides commands \textsmalle, \textlarger
%\usepackage{xspace} % gentle spacing in macros, such as \newcommand{\acims}{\textsc{acim}s\xspace}

% Page formatting utility:
%\usepackage{geometry}

\usepackage{booktabs}   %% For formal tables:
                        %% http://ctan.org/pkg/booktabs
\usepackage[labelformat=simple]{subcaption} %% For complex figures with subfigures/subcaptions
                        %% http://ctan.org/pkg/subcaption
% Options to subcaption are to label and refer to subfigures as Fig 1(a) etc.
\renewcommand\thesubfigure{(\alph{subfigure})}

\usepackage[T1]{fontenc} % needed for scaling fancy fonts (?)
\usepackage[utf8]{inputenc} % not sure this is needed

\usepackage{amssymb}
%\usepackage[table]{xcolor}

% For code
\usepackage[final]{listings}
\lstset{mathescape=true}

% For code highlighting
% \usepackage{bold-extra}

% Tikz
\usepackage{tikz}
\usetikzlibrary{matrix,arrows,positioning,calc,fit,backgrounds}

% To control enum item labelling/numbering
\usepackage[shortlabels, inline]{enumitem}
% To give custom item labels and reference them
\makeatletter
\newcommand{\myitem}[1][]{
  \protected@edef\@currentlabel{#1}%
\item[#1]
}
\makeatother

% To stop aligned env swallowing up []s
\usepackage{mathtools}

% To use ifstrempty
\usepackage{etoolbox}

% For math mode tables
\usepackage{array}
% A text column in array
\newcolumntype{L}{>$l<$}

% For \llbracket and \rrbracket
\usepackage{stmaryrd}

% For dashed boxes
\usepackage{dashbox}

% For big separating conjunction
\usepackage{scalerel}

% For mathpar environment
\usepackage{mathpartir}

\usepackage{xspace}
\usepackage{multirow}

% To stop citations overflowing lines
\usepackage{breakcites}

% For citet command
\usepackage{natbib}
\setcitestyle{%
    authoryear,%
    open={[},close={]},citesep={;},%
    aysep={},yysep={,},%
    notesep={, }}
\let\cite\citep

%%
%% Place here your \newtheorem's:
%%

\theoremstyle{plain}
\newtheorem{theorem}{Theorem}[chapter]
\newtheorem{conjecture}[theorem]{Conjecture}
\newtheorem{proposition}[theorem]{Proposition}
\newtheorem{lemma}[theorem]{Lemma}
\newtheorem{corollary}[theorem]{Corollary}
\theoremstyle{definition}
\newtheorem{example}[theorem]{Example}
\newtheorem{definition}[theorem]{Definition}
\theoremstyle{plain}


% ----------------------------------------
% Generic definitions
% ----------------------------------------
% Required packages: listings, tikz

% A footnote without a marker
\newcommand\blfootnote[1]{%
  \begingroup
  \renewcommand\thefootnote{}\footnote{#1}%
  \addtocounter{footnote}{-1}%
  \endgroup
}

\renewcommand{\le}{\leqslant}
\renewcommand{\ge}{\geqslant}
% \renewcommand{\emptyset}{\ensuremath{\varnothing}}
% \newcommand{\ds}{\displaystyle}

% ----------------------------------------
% Paper specific macros & commands
% ----------------------------------------


% Put your definitions here


%%% Local Variables:
%%% mode: latex
%%% TeX-master: "thesis"
%%% End:

\usepackage{color}
\usepackage{amssymb}
\usepackage{amsmath}
\usepackage{url}
\usepackage[utf8]{inputenc}
\usepackage{appendix}
\usepackage{pifont}
\usepackage{algorithm2e}
\usepackage[normalem]{ulem}


\renewcommand{\topfraction}{0.9}     % max fraction of floats at top
\renewcommand{\bottomfraction}{0.8}     % max fraction of floats at bottom
\setcounter{topnumber}{4}
\setcounter{bottomnumber}{4}
\setcounter{totalnumber}{8}     % 2 may work better
\setcounter{dbltopnumber}{4}    % for 2-column pages
\renewcommand{\dbltopfraction}{0.95}     % fit big float above 2-col. text
\renewcommand{\textfraction}{0.07}     % allow minimal text w. figs
\renewcommand{\floatpagefraction}{0.7}     % require fuller float pages
\renewcommand{\dblfloatpagefraction}{0.7}     % require fuller float pages

\usepackage{microtype}
\usepackage{wrapfig}
\usepackage{minted}

%%% Math Symbols

%%% prism def.
\newcommand{\T}{\mathcal{T}}
\renewcommand{\P}{\mathcal{P}}
\renewcommand{\S}{\mathcal{S}}
\newcommand{\M}{\mathcal{M}}
\newcommand{\V}{\mathcal{V}}
\newcommand{\Tet}{\boldsymbol{T}}
\renewcommand{\H}{\mathcal{H}}
\newcommand{\C}{\mathcal{C}}
\newcommand{\N}{\mathcal{N}}
\newcommand{\Prism}{\Delta}
\newcommand{\D}{\mathcal{D}}
\newcommand{\J}{\boldsymbol{J}}

%%scaf


% section for terminology in method section
\DeclareMathOperator{\Mesh}{\mathcal{M}}

% \DeclareMathOperator{\V}{\mathcal{V}}
\DeclareMathOperator{\F}{\mathcal{F}}
\DeclareMathOperator{\ScaF}{\mathcal{S}}

\DeclareMathOperator{\matV}{\mathbf{V}}
\DeclareMathOperator{\matF}{\mathbf{F}}
\DeclareMathOperator{\matS}{\mathbf{S}}
\DeclareMathOperator{\matU}{\mathbf{U}}

\DeclareMathOperator{\Distort}{\mathcal{D}}

\DeclareMathOperator{\R}{\varmathbb{R}}

\DeclareMathOperator*{\argmin}{argmin}
\DeclareMathOperator{\round}{round}
\newcommand{\norm}[1]{\lVert #1 \rVert}


%%% bichon def
% \newcommand{\M}{\mathcal{M}}
% \newcommand{\T}{\mathcal{T}}
\newcommand{\B}{\mathcal{B}}
% \renewcommand{\P}{\mathcal{P}}
% \newcommand{\F}{\mathcal{F}}
% \newcommand{\V}{\mathcal{V}}
\newcommand{\PS}{\overline{\mathcal{S}}}
\newcommand{\prS}{\widetilde{\mathcal{S}}}
\newcommand{\ps}{projection shell}
% \newtheorem{proposition}[theorem]{Proposition}

% \DeclareMathOperator*{\argmin}{arg\,min}
\DeclareMathOperator*{\tr}{tr}
\DeclareMathOperator*{\isinside}{is\_inside}
\DeclareMathOperator*{\issection}{is\_section}
\usepackage{adjustbox}

% \newcommand{\listappendicesname}{Appendices}
% \newlistof{appendices}{apc}{\listappendicesname}
% \newcommand{\appendices}[1]{\addcontentsline{apc}{appendices}{#1}}


\newenvironment{longlisting}{\captionsetup{type=listing}}{}



% ----------------------------------------
% Document header
% ----------------------------------------

%% Cross-referencing utilities. Use one or the other--whichever you prefer--
%% but comment out both lines for final version.
%\usepackage{showlabels}
%\usepackage{showkeys}

\begin{document}
%% Produces a test "layout" page, for "debugging" purposes only.
%% Comment out for final version.
%\layout % requires package layout (see above, on this same file)


%%%%%% Title page %%%%%%%%%%%
%% Sets page numbering to "roman style" i, ii, iii, iv, etc:
\pagenumbering{roman}
%
%% No numbering in the title page:
\thispagestyle{empty}
%
\vspace*{25pt}
\begin{center}
  {\Large
    \begin{doublespace}
      {\textcolor{SchoolColor}{\textsc{\thesistitle}}}
    \end{doublespace}
  }
  \vspace{.7in}

  by
  \vspace{.7in}

  \thesisauthor
  \vfill

  \begin{doublespace}
    \textsc{
    A dissertation submitted in partial fulfillment\\
    of the requirements for the degree of\\
    Doctor of Philosophy\\
    Department of \thesisdept\\
    New York University\\
    \gradmonth, \gradyear}
  \end{doublespace}
\end{center}
\vfill

\noindent\makebox[\textwidth]{\hfill\makebox[2.5in]{\hrulefill}}\\
\makebox[\textwidth]{\hfill\makebox[2.5in]{\hfill\thesisadvisor}}

\newpage


%%%%%%%%%%%%% Copyright page %%%%%%%%%%%%%%%%%%
\thispagestyle{empty}
\vspace*{25pt}
\begin{center}
  \scshape \noindent \small \copyright \  \small  \thesisauthor \\
  all rights reserved, \gradyear
\end{center}
\vspace*{0in}
\newpage


%%%%%%%%%%%%%% Dedication %%%%%%%%%%%%%%%%%
%% Comment out the following lines if you do not want to dedicate
%% this to anyone...
\cleardoublepage
\phantomsection
\addcontentsline{toc}{chapter}{Dedication}
\vspace*{\fill}
\begin{center}
  \thesisdedication
\end{center}
\vfill
\newpage


%%%%%%%%%%%%%% Acknowledgements %%%%%%%%%%%%
%% Comment out the following lines if you do not want to acknowledge
%% anyone's help...
\chapter*{Acknowledgements}
\addcontentsline{toc}{chapter}{Acknowledgments}

% I am priviledges to be able to finish the work, in the turbulent time.
Working remotely from my home.

Something about COVID.

\begin{itemize}
  \item nTopology Inc, Suraj, support.
  \item Adobe: Vova
  \item Mentor: Daniele, Denis, 
  \item Collaborators: Scott, Teseo, 
  \item Labmates: Francis, Francisca, Chase, Julian, Qingnan, Davi, Yixin, YunFan, Jiacheng, Jeremie
  \item Officemates: Junbo, Xiang Z, Xifeng G, 
  \item Staff: Shenglong, Hong
  \item Friends and family: 
\end{itemize}


\lipsum[1-2]

\newpage


%%%% Abstract %%%%%%%%%%%%%%%%%%
\chapter*{Abstract}
\addcontentsline{toc}{chapter}{Abstract}

Various applications, from artistic creation, to scientific computing, require the processing and reasoning of 3D digital objects.
The computational modeling of 3D geometric shapes, materials, and textures, as well as the simulation of their deformation and interactions, is essential to bring the algorithmic power of computing to real-life manufacture, architecture, and medical device design.
Depending on the specific numerical properties, better algorithm designs might prefer 3D data with different representations, for example, in planes, surfaces, or inside volumes.
This thesis investigates the problem related to the representations of data on 3D shapes and across different domains,
so computations for different stages within a pipeline, may come together synergistically without manual tuning that disrupts an automated data flow.
I propose novel geometrical principles in various geometric modeling and processing stages.
I also showcase various geometric computing applications that easily integrate such principles to guarantee the geometry validity and algorithm effectiveness of surface parameterization, rendering, deformation/animation, and mechanical simulation.
In addition, we can finally explore creative solutions that reliably coarsen the surface. Such simplification accelerates everyday geometric modeling operations; the contribution also includes a scalable method to construct coarse and curved meshes for fast animation and scientific computing.
Furthermore, the thesis provides a declarative way to formulate mesh processing and adaptation algorithms to facilitate the practical development of robust and reliable mesh processing software.
Finally, the thesis includes extensive numerical validations involving tens of thousands of complex geometry shapes. %And to maintain replicability and foster further research in this direction, I also released the implementation and generated data to be open source and accessible.


\newpage


%%%% Table of Contents %%%%%%%%%%%%
\tableofcontents


%%%%% List of Figures %%%%%%%%%%%%%
%% Comment out the following two lines if your thesis does not
%% contain any figures. The list of figures contains only
%% those figures included within the "figure" environment.
\cleardoublepage
\phantomsection
\addcontentsline{toc}{chapter}{List of Figures}
\listoffigures
\newpage


%%%%% List of Tables %%%%%%%%%%%%%
%% Comment out the following two lines if your thesis does not
%% contain any tables. The list of tables contains only
%% those tables included within the "table" environment.
\cleardoublepage
\phantomsection
\addcontentsline{toc}{chapter}{List of Tables}
\listoftables
\newpage


%%%%% Body of thesis starts %%%%%%%%%%%%
\pagenumbering{arabic} % switches page numbering to arabic: 1, 2, 3, etc.


% ----------------------------------------
% Body of Thesis
% ----------------------------------------

\chapter{Introduction}
\label{chp-introduction}

% % background, modeling and simulation are connected through complex shape
% 

% We make our objects with manufacture pipeline, starting from a digital engineering design.
% advanced manufacturing and design techniques more and more complex shapes (engineering design).
Most of the everyday objects that we invent and interact in the physical world, ranging from electronic devices to transportation vehicles, are manufactured through a digital aided production pipeline.
First, given a desirable functionality, the designer compose a shape, then prototype and test the object under specified physical conditions. e.g. stress, or heat, or fluidic. 
Therefore, automatic computer algorithms for the design and fabrication would ensure the reliability and accessibility of medical devices. To do so requires a clear understanding of physical simulation and geometry design.

% On the other hand
On the other hand, composition of digital assets are through the similar pipeline, including human avatars, or objects that we are growing essential for the  virtual environment, thanks for 3D scanning techniques to obtain a digital shape, and then perform editing, fairing, modifications and other modeling operations. larger and larger quantites, and versatile of organic shapes beyond bunny. (scanning) and we want the dynamic deformation/collision response, in the information rich virtual environment, for interaction and physics based deformation.

% In both sides of these processes, there is the problem of robustness. And the need to connect modeling with simulation. and to process the information back and forth.
Researchers usually examine the computational design and physical simulation separately, leading to different computational tools for the discretized geometry. Different representations between design methods and various adaptations of simulations, even for the same shape, are not reliably correlated. In the general case, the material, attribute, or experiment setting cannot be robustly re-used on another representation without laborious manually fixing. Such a pipeline demands the engineer or designer to understand precisely the properties of the specific numerical methods of choice. It also prohibits an automatic and robust pipeline from optimizing the desired physical durability or permeability structure. 

\fbox{\parbox{0.9\textwidth}{\textcolor{gray}{
This dissertation is primarily concerned with the problem of how to effectively dealing with attributes, in various scenarios.
\begin{itemize}
  \item SCAF: representations to store attributes in plane, and applying such techniques to volume parameterization. This is crucial step for reliable attribute transfer between surfaces and volumes. color texture materials.
  \emph{Robust}, \emph{Deformation}, \emph{Geometry}.
\item SHELL: building on top, novel parameterization strategies between surfaces, for complex geometry, with nontrivial topology. \emph{Robust}. \emph{Simulations}, \emph{Geometry}. 
\item CURVE: application of shell, transfer serving robust simulation. \emph{Robust}. \emph{Simulations}, \emph{Geometry}. 
\item WMTK: speed up, robustness, aforementioned conditions, easily integrate constraints into new apps. \emph{Robust}
\end{itemize}
Some geometrical constraints suitable for computation, that guarantees properties important to various aspect of design and simulation.
}}}

Through this thesis, I investigate, develop, and implement computational principle to relate information across digital design and simulation pipelines reliably. 
The central technique is to leverage auxiliary geometric constructions to introduce novel constraints. 
By integrating such geometric constraints into the parameterization, or mesh optimization pipeline,  I am able to achieve provable guarantees. 
At the same time, geometric algorithms are often troubled with numerical problems. For this, I perform extensive experiments to back up in real.

I will first introduce the method of SCAF, 
augment triangle with virtual doamin method.

\emph{Synergy.}

I showcased reliable, efficient, and robust simulations from an arbitrarily complex design setup using the principle. After establishing such connections, we are now ready to systematically evaluate and improve the computational bottlenecks on the entire engineering pipeline.
The central idea in this dissertation is to develop geometric computing tools that reliably associate different representations of discrete geometry entities at diverse design and simulation stages. Instead of re-assigning properties at various phases of the design-simulation pipeline, or whenever the required budget is different, a good association between geometry at different stages gives more freedom to the design and development of automatic and adaptive solvers. The adaptation can be bi-directional: on the one hand, with coarse and concise tessellation, the algorithm utilizes the information associated with the original design to obtain more faithful simulations. On the other hand, in early-stage prototyping and automatic data generation for deep learning, the principles established to simplify the shape and retain the settings provide a fast answer. 

%
Next, the reliable transfer between different representations allows for the encapsulation of the simulation process and promotes the adoption of advanced finite element methods from the scientific computing community. As a notable example, high order tetrahedral meshes accurately approximate a given computation domain, with much fewer degrees of freedom. These meshes enjoy orders of magnitude speed-up with a similar simulation setup compared to their more commonly used linear counterparts in a range of physical simulations, including solid mechanics and fluid dynamics. Is proposed the first robust algorithm that can reliably convert extensive complex geometry models (as surface triangle mesh) to valid and coarse high order tetrahedral meshes within specified geometry tolerance. The method is validated on various complex geometry, including mechanical parts, biological reconstructions, and microstructures. Additionally, the results are reliably associated with their input, making it an ideal substitute for simulation workload, with all the information kept intact. 
%The method is published in “Bijective and Coarse High-Order Tetrahedral Meshes,” Jiang et al., ACM Transaction on Graphics, 2021:


% Contributions goes with related works.
% Mathematical Contributions:
% we extend the computational fronts of bijective maps across surfaces, and computational vector fields

% Algorithmic contributions:
% Robust algorithms through experimental and theory.

% Educational:
% auxiliary line or helping line is a useful tool in elementry geometry proofs. It is usually overlooked in more advanced textbooks. However, as I have showcased in several applications, educations with such tools would make sense computationally.
% 
Outreach and impact:

% 
The rest of the thesis is structured as follows,
chapter 2 discuss our contributions and related works.
chapter 3 introduce bijective parameteriazations with triangulation augmentation, as a technique to speed up and guarantee the computation of bijective maps in 2d and 3D, as an crucial intermediate step for registering, storing, and processing surface and volume signals.
Chapter 4 transitions into shell map, a novel surface and volume mapping technique, that is suitable for attribute handlign during large scale robust processing. 
Chapter 5 further complete the technique in Chapter 4, and discuss , along the way, we showcase an important application: how to use the shell to generate conforming tetrahedral meshes, and high order curve meshes, which is promising building block in new paradigm of scalable physical simulations. 
While being useful in our various applications, such geometry constructions remains out of reach to be plugin into many existing algorithms and implementations, we abstract the problem, and introduce an new specification paradigm in Chapter 6, as a algorithm developement toolkit, aiming to make our contribution accesible for more researchers in the field between robust geometry modeling, and scalable physical simulation.

% Sample content:
\lipsum[1-2]

\begin{figure}
  \centering
  \begin{tikzpicture}[scale=3]
    \draw[step=.5cm, gray, very thin] (-1.2,-1.2) grid (1.2,1.2); 
    \filldraw[fill=green!20,draw=green!50!black] (0,0) -- (3mm,0mm) arc (0:30:3mm) -- cycle; 
    \draw[->] (-1.25,0) -- (1.25,0) coordinate (x axis);
    \draw[->] (0,-1.25) -- (0,1.25) coordinate (y axis);
    \draw (0,0) circle (1cm);
    \draw[very thick,red] (30:1cm) -- node[left,fill=white] {$\sin \alpha$} (30:1cm |- x axis);
    \draw[very thick,blue] (30:1cm |- x axis) -- node[below=2pt,fill=white] {$\cos \alpha$} (0,0);
    \draw (0,0) -- (30:1cm);
    \foreach \x/\xtext in {-1, -0.5/-\frac{1}{2}, 1} 
    \draw (\x cm,1pt) -- (\x cm,-1pt) node[anchor=north,fill=white] {$\xtext$};
    \foreach \y/\ytext in {-1, -0.5/-\frac{1}{2}, 0.5/\frac{1}{2}, 1} 
    \draw (1pt,\y cm) -- (-1pt,\y cm) node[anchor=east,fill=white] {$\ytext$};
  \end{tikzpicture}
  \caption{A pictorial view of \refThm{thm-pythagorean}.}
  \label{fig-pythagorean}
\end{figure}

\begin{definition}
  A function $f$ is said to be \emph{continuous} if its derivative exists at every point.
\end{definition}

\begin{lemma}
  Let $f$ be a function whose derivative exists in every point, then $f$ is 
  a continuous function.
\end{lemma}

\begin{theorem}[Pythagorean theorem]
  \label{thm-pythagorean}
  This is a theorema about right triangles and can be summarised in the next 
  equation 
  \[ x^2 + y^2 = z^2 \]
\end{theorem}
\begin{proof}
  I have discovered a truly marvelous proof of this, which this margin is too narrow to contain.
\end{proof}

And a consequence of \refThm{thm-pythagorean} is the statement in the next 
corollary~\cite{lamport94}.

\begin{corollary}
  There's no right rectangle whose sides measure 3cm, 4cm, and 6cm.
\end{corollary}

\lipsum[3-5]

\begin{table}
  \centering
  \caption{Predicted final standings of Group B.}
  \begin{tabular}{l*{6}{c}r}
    Team              & P & W & D & L & F  & A & Pts \\
    \hline
    Manchester United & 6 & 4 & 0 & 2 & 10 & 5 & 12  \\
    Celtic            & 6 & 3 & 0 & 3 &  8 & 9 &  9  \\
    Benfica           & 6 & 2 & 1 & 3 &  7 & 8 &  7  \\
    FC Copenhagen     & 6 & 2 & 1 & 3 &  5 & 8 &  7  \\
  \end{tabular}
  \label{tab-forecast}
\end{table}

\lipsum[5-7]

\chapter{Preliminaries}
\label{chp-preliminaries}

% \input{preliminaries}
\lipsum

\chapter{Proof of the Reimann Hypothesis}
\label{chp-proof}

% \input{proof}
\lipsum

\chapter{Conclusion}
\label{chp-conclusion}

% In the previous chapters, the dissertation has discussed several problems related to the bijective mappings between different spatial domains and introduced relevant computational principles to guarantee the validity of attribute association and present robust algorithms to process information across digital shapes efficiently. 
In this concluding chapter, I argue that the significance of these principles lies not only during the robust and scalable processing of digital geometry shapes but also in developing the next-generation pipeline of geometric modeling and physical simulation. 
To achieve this broader goal, there are still many open and exciting problems to solve.

\begin{enumerate}
  \item \emph{Shape-Aware Adaptive Mesh Refinement}:
In chapter~\ref{chp:curve}, we showcased the benefit of surface and volume simplification in the context of scalable visualization and efficient physical simulation. 
On the other hand, we can further extend the pipeline to refine back towards a given geometry by adding detail and smoothness adaptively. Combined with an Adaptive Mesh Refinement (AMR) pipeline, such features are essential for scientific computing and computational design.
Specifically, given a detailed reference shape with material assignment, the algorithm in chapters~\ref{chp:shell, chp:curve} coarsen the volume domain with drastically fewer elements. 
The governing equations with transferred boundary conditions may be efficiently solved in such a simplified domain. The geometry might refine where needed depending on the current solution's smoothness or accuracy. 
Note that such refinement should always converge to the reference shape to bring the geometry error down, and finally, we can achieve an adaptive solution on an otherwise dense geometry. 

\item \emph{Bijective Shell in the Wild} % partial shell parsely
In chapter~\ref{chp:shell}, the bijective shell is formulated on manifold meshes, 
to guarantee strictly bijective attribute transfer everywhere.
However, a blended framework with envelope \cite{hu2018tetrahedral, Wang:2021} can open the door to partially corrupted or occluded data. 
Such characteristics are common in 3d scanned datasets or MRI medical datasets, and point cloud and implicit represented surface data may all be processed within the same framework.
Given an input mixed shape description, the algorithm would first segment the partial regions with attributes and apply the bijective shell construction algorithm on the partial surface. During downstream remeshing, different parts would be checked with varying envelope criteria, and extra care should be taken to ensure continuity across the segmentation boundaries. 

\item \emph{High-Order reduced FEM simulations}
The high order finite element method is a concept already with decades of development. Still, we believe that the availability of more high-order surface and volume meshes (chapter~\ref{chp:curve}) would benefit its adoption and release the potential inefficient physical computations. In addition, resolving frictional contact with curved representations, \cite{ferguson2022high} can make dynamic simulations more scalable and efficient. 
While surface modeling software often uses curved geometries (e.g. B{\'e}zier and NURBS), the conversion of these geometries to curved simulation meshes is still an open problem, and existing solution are not robust. In this aspect, our contribution of robust and intersection-free curve surface modeling can lead to a more systematically robust industrial design and simulation pipeline.

\item \emph{Improved Computational Fabrication of Customized Devices}
The chapters~\ref{chp:shell,chp:curve} have introduced the robust association between design and simulation representations. A closely related concept with attribute transfer is differentiability. Novel schemes for differentiable physics on the same mesh has been proposed. And in the presence of large deformation with connectivity change, our construction comes in handy by bijectively associating through the remeshing process. 
With such end-to-end associations, the initial design decisions in the shape space can be reflected in the final durability of the parts, and differentiability ensures that the design can be instructed to improve automatically.  

\item \emph{Volume-Aware 3D Geometry Processing}
can be another venue for interesting future works. 
Similar to the principle we explained in chapter~\ref{chp:scaf}, a background mesh provides implicit and easy ways to check for the geometry validity. For example, the scaffold mesh converts global surface self-intersection into locally inversion checks. It has the potential to unify the framework in different shapes and orders of finite element cells, and is partially adopted in the construction of curved shell~\ref{chp:curve}. Traditional geometry processing algorithms, including smoothing, fairing, and animation, can be more efficient and reliable under topology constraints with a tetrahedral-based virtual domain. A notable example is the surface insertion or mesh arrangement problem we used in Chapter~\ref{chp:wmtk}: the exact version is still the bottleneck of our tetrahedral mesh generation algorithm, but recent advances in mesh arrangement~\cite{Hu:2019:fTetWild,zhou2016mesh,ember2022} are promising on this path.

\item \emph{Robust Large-Scale Geometry Dataset Processing}
Finally, advances in new processing algorithms are necessary for improved and automatic data collection techniques. 
Better data is the stepping stone to fully leveraging information from the large-scale dataset\cite{Koch_2019_CVPR,zhou2016thingi10k}. 
More significant amounts of data pave the way for more tasks, including semantic shape segmentation, point cloud feature detection, object physics with interaction understanding, and robotics manipulation.

\end{enumerate}


\lipsum


%% If your thesis has different "Parts", use commands such as the following:
%\part{First Part\label{part:one}}%
% \input{chap1}
%\input{chap2} % further chapters -- change file names to meaningful things...
%\input{chap3}
%\part{Second Part\label{part:two}}%
%\input{chap4}
%\input{chap5}
%\input{chap6}


%%%%% Appendices start %%%%%%%%%%%%%%%%
%% Comment out the following if your thesis has no appendix

\appendix

\chapter{Appendix}

% \input{appendix}
\lipsum

%% Note: If your thesis has more than one appendix, NYU requires a "list of
%% appendices" page before the body of the thesis. I don't provide the tools
%% to create that here, so you're on your own for that one... Sorry.


%%%% Input bibliography file %%%%%%%%%%%%%%%
%% For computer science dissertations, I'd recommend using the bibly package
%% to automatically create the .bib file from your citations:
%% https://github.com/michael-emmi/bibly

\cleardoublepage
\phantomsection
\bibliographystyle{apalike}
\addcontentsline{toc}{chapter}{Bibliography}

% \bibliography{dblp,references}

% The following is just for the sample template,
% I'd recommend deleting this and using the \bibliography command above
\begin{thebibliography}{99}
\bibitem[Lamport, 1994]{lamport94}
  Leslie Lamport,
  \textit{\LaTeX: a document preparation system},
  Addison Wesley, Massachusetts,
  2nd edition,
  1994.
\end{thebibliography}


\end{document}

%%% Local Variables:
%%% mode: latex
%%% TeX-master: t
%%% End:
