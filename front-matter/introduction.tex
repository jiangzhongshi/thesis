% background, modeling and simulation are connected through complex shape
% 

% We make our objects with manufacture pipeline, starting from a digital engineering design.
% advanced manufacturing and design techniques more and more complex shapes (engineering design).
Most of the everyday objects that we invent and interact in the physical world, ranging from electronic devices to transportation vehicles, are manufactured through a digital aided production pipeline.
First, given a desirable functionality, the designer compose a shape, then prototype and test the object under specified physical conditions. e.g. stress, or heat, or fluidic. 
Therefore, automatic computer algorithms for the design and fabrication would ensure the reliability and accessibility of medical devices. To do so requires a clear understanding of physical simulation and geometry design.

% On the other hand
On the other hand, composition of digital assets are through the similar pipeline, including human avatars, or objects that we are growing essential for the  virtual environment, thanks for 3D scanning techniques to obtain a digital shape, and then perform editing, fairing, modifications and other modeling operations. larger and larger quantites, and versatile of organic shapes beyond bunny. (scanning) and we want the dynamic deformation/collision response, in the information rich virtual environment, for interaction and physics based deformation.

% In both sides of these processes, there is the problem of robustness. And the need to connect modeling with simulation. and to process the information back and forth.
Researchers usually examine the computational design and physical simulation separately, leading to different computational tools for the discretized geometry. Different representations between design methods and various adaptations of simulations, even for the same shape, are not reliably correlated. In the general case, the material, attribute, or experiment setting cannot be robustly re-used on another representation without laborious manually fixing. Such a pipeline demands the engineer or designer to understand precisely the properties of the specific numerical methods of choice. It also prohibits an automatic and robust pipeline from optimizing the desired physical durability or permeability structure. 

\fbox{\parbox{0.9\textwidth}{\textcolor{gray}{
This dissertation is primarily concerned with the problem of how to effectively dealing with attributes, in various scenarios.
\begin{itemize}
  \item SCAF: representations to store attributes in plane, and applying such techniques to volume parameterization. This is crucial step for reliable attribute transfer between surfaces and volumes. color texture materials.
  \emph{Robust}, \emph{Deformation}, \emph{Geometry}.
\item SHELL: building on top, novel parameterization strategies between surfaces, for complex geometry, with nontrivial topology. \emph{Robust}. \emph{Simulations}, \emph{Geometry}. 
\item CURVE: application of shell, transfer serving robust simulation. \emph{Robust}. \emph{Simulations}, \emph{Geometry}. 
\item WMTK: speed up, robustness, aforementioned conditions, easily integrate constraints into new apps. \emph{Robust}
\end{itemize}
Some geometrical constraints suitable for computation, that guarantees properties important to various aspect of design and simulation.
}}}

Through this thesis, I investigate, develop, and implement computational principle to relate information across digital design and simulation pipelines reliably. 
The central technique is to leverage auxiliary geometric constructions to introduce novel constraints. 
By integrating such geometric constraints into the parameterization, or mesh optimization pipeline,  I am able to achieve provable guarantees. 
At the same time, geometric algorithms are often troubled with numerical problems. For this, I perform extensive experiments to back up in real.

I will first introduce the method of SCAF, 
augment triangle with virtual doamin method.

\emph{Synergy.}

I showcased reliable, efficient, and robust simulations from an arbitrarily complex design setup using the principle. After establishing such connections, we are now ready to systematically evaluate and improve the computational bottlenecks on the entire engineering pipeline.
The central idea in this dissertation is to develop geometric computing tools that reliably associate different representations of discrete geometry entities at diverse design and simulation stages. Instead of re-assigning properties at various phases of the design-simulation pipeline, or whenever the required budget is different, a good association between geometry at different stages gives more freedom to the design and development of automatic and adaptive solvers. The adaptation can be bi-directional: on the one hand, with coarse and concise tessellation, the algorithm utilizes the information associated with the original design to obtain more faithful simulations. On the other hand, in early-stage prototyping and automatic data generation for deep learning, the principles established to simplify the shape and retain the settings provide a fast answer. 

%
Next, the reliable transfer between different representations allows for the encapsulation of the simulation process and promotes the adoption of advanced finite element methods from the scientific computing community. As a notable example, high order tetrahedral meshes accurately approximate a given computation domain, with much fewer degrees of freedom. These meshes enjoy orders of magnitude speed-up with a similar simulation setup compared to their more commonly used linear counterparts in a range of physical simulations, including solid mechanics and fluid dynamics. Is proposed the first robust algorithm that can reliably convert extensive complex geometry models (as surface triangle mesh) to valid and coarse high order tetrahedral meshes within specified geometry tolerance. The method is validated on various complex geometry, including mechanical parts, biological reconstructions, and microstructures. Additionally, the results are reliably associated with their input, making it an ideal substitute for simulation workload, with all the information kept intact. 
%The method is published in “Bijective and Coarse High-Order Tetrahedral Meshes,” Jiang et al., ACM Transaction on Graphics, 2021:


% Contributions goes with related works.
% Mathematical Contributions:
% we extend the computational fronts of bijective maps across surfaces, and computational vector fields

% Algorithmic contributions:
% Robust algorithms through experimental and theory.

% Educational:
% auxiliary line or helping line is a useful tool in elementry geometry proofs. It is usually overlooked in more advanced textbooks. However, as I have showcased in several applications, educations with such tools would make sense computationally.
% 
Outreach and impact:

% 
The rest of the thesis is structured as follows,
chapter 2 discuss our contributions and related works.
chapter 3 introduce bijective parameteriazations with triangulation augmentation, as a technique to speed up and guarantee the computation of bijective maps in 2d and 3D, as an crucial intermediate step for registering, storing, and processing surface and volume signals.
Chapter 4 transitions into shell map, a novel surface and volume mapping technique, that is suitable for attribute handlign during large scale robust processing. 
Chapter 5 further complete the technique in Chapter 4, and discuss , along the way, we showcase an important application: how to use the shell to generate conforming tetrahedral meshes, and high order curve meshes, which is promising building block in new paradigm of scalable physical simulations. 
While being useful in our various applications, such geometry constructions remains out of reach to be plugin into many existing algorithms and implementations, we abstract the problem, and introduce an new specification paradigm in Chapter 6, as a algorithm developement toolkit, aiming to make our contribution accesible for more researchers in the field between robust geometry modeling, and scalable physical simulation.