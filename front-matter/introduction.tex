Most of the objects that we invent and interact with from the physical world, ranging from electronic devices to vehicles, are manufactured through a digitally aided production pipeline.
In practice, a typical product begins its life-cycle from a digital specification of the shape and material, 
and we can take advantage of the numerical solution of partial differential equations (PDEs) to portray how a design would behave even before it is created.
Computer simulation would then test the behavior to match the designer's intent: a car hood's deformation under the pressure of collision or a steel pan's temperature distribution when heated on a stove. In this case, a \emph{robust}, \emph{efficient} and \emph{accurate} design-simulation pipeline could lift many iterations of manufacturing to the digital platform. In addition, as it is safer to conduct more stress experiments in a virtual environment, a reliable simulation also promotes safety, especially in designing bio-mechanical equipment.

On the other hand, we also create digital assets from real objects through an inverse pipeline. Traditionally, it serves as a way to reverse engineer a part into a digital format or preserve ancient or otherwise valuable artifacts. And more recently, with the advance of 3D scanning and capture devices, people are increasingly interested in representing themselves and their surroundings as digital avatars and virtual environments. Such techniques typically go from raw scanning depth images to obtaining a digital 3D shape. The users and algorithms may then perform simplification, editing, fairing, modifications, or other modeling operations. The digital objects will then interact, deform, and collide with each other in a virtual modeling or game environment, abiding by essential physics.
Such advances give us growing quantities of 3D model datasets, with great versatility of organic shapes, as well as their defects, including holes, self-overlapping surfaces, and so on. Such defects require expertise and manual labor to fix, throttling the effort for everyone to enjoy and create in the virtual world. To build an information-rich virtual environment, it is also essential for us to extend and scale up the algorithmic capabilities to process and understand the shape, material, and physics of these shapes.

There are various approaches to specify a shape, depending on the different geometry characteristics. Shape modeling evolves beyond smooth and regular objects like sculpture or furniture. Algorithm-deduced structures have gained popularity since they deliver the promise of maximal durability with limited material. Downstream, computer simulation requires a discrete tessellation of the object. A notable example is a mesh: a data structure containing a list of vertices for the spatial locations, a list of edges connecting different vertices, and a list of simple polygons or polyhedra. However, it is rarely the case that one mesh fits all. More degrees of freedom give a more accurate description of the desired shape, but it inevitably means more computational resources are needed to handle the computation.

Researchers usually examine the 3D acquisition, computational design, deformable animation, and physical simulation separately. Different approaches lead to different computational tools for discretized geometry.
The representations between design methods and various adaptations of simulations, even for the same shape, are not reliably connected.
In the general case, the material, attribute, or experiment setting cannot be robustly re-used on another representation without laborious manually fixing. Such a pipeline demands the engineer or designer to understand precisely the properties of the specific numerical methods of choice. It also prohibits an automatic and robust pipeline from optimizing the desired physical durability or permeability structure.

In contrast to the 2D representations of images, which are commonly stored on grid structures and processed with standard matrix operations, there have been different candidates to represent and process the information on a 3D shape with various representations. 
Ranging from multi-view depth images, point clouds, structured/unstructured meshes, and there are different trade-offs associated, ranging from memory efficiency, and specialized algorithms. And across different stages of the computational pipeline regarding a 3D shape, different representations might be used, to adapt to specialized algorithms: for example, 3D acquisition devices produce multi-view depth images; some design and modeling tools make use of the topology-flexible implicit surfaces as the representation for lattice or porous structures; digital asset modeling tool-chains and industry make use of surface triangle or quadrilateral meshes extensively; physical simulation requires a spatial partitioning, as commonly tetrahedral meshes to carry out the quantities inside a volume. 

The thesis aims to develop computational principles to reliably associate different representations of discrete geometry entities at diverse design and simulation stages. Instead of re-assigning properties at various phases of the design-simulation pipeline, or whenever the required budget is different, a good association between geometry at different stages gives more freedom to the design and development of automatic and adaptive solvers. The adaptation can be bi-directional: on the one hand, with coarse and concise tessellation, the algorithm utilizes the information associated with the original design to obtain more faithful simulations. On the other hand, in early-stage prototyping, and automatic data generation for deep learning, the principles I established allow for simplifying the shape and retaining the settings, providing a fast answer. In addition, I use the same principle to design the first large-scale and robust approach for curved mesh representation, which provides accurate representation as well as efficient physical simulations.
In addition, by reliably connecting the data across different representations and at different stages, guarantees of the correctness of properties can be safely upheld, which would reduce the chance of mistakes and iterations by manual labor. 
Furthermore,  since different algorithms perform well on different representations, multiple algorithms can finally cofunction, and synergistically improve the performance of an overall industry pipeline.
Such a contribution outreach to 3D model design and acquisition pipeline, as well as their physical behavior predication, and manufactured.

The thesis investigates several problems and studies the techniques relevant to meshes and their associated attributes, and the central technique is to leverage auxiliary geometric constructions to design novel constraints, usable in the framework of mathematical optimization. 
By integrating such geometric constraints into the surface parameterization, or mesh generation and optimization pipeline, I am able to arrive at provable guarantees that the meshes are good candidates for downstream applications, where the attributes can be robustly transferred.
In the meantime, with the limited precision implementations, geometric algorithms are often troubled with numerical problems. Thus, I also perform extensive experiments to back up the claim and make the algorithm's concept more easily accessible to industrial applications.

I showcased reliable, efficient, and robust simulations from an arbitrarily complex design setup using the principle. After establishing such connections, we are now ready to systematically evaluate and improve the computational bottlenecks in the entire engineering pipeline.
The central idea of this dissertation is to develop geometric computing tools that reliably associate different representations of discrete geometry entities at diverse design and simulation stages. 
Instead of re-assigning properties at various phases of the design-simulation pipeline, or whenever the required budget is different, a good association between geometry at different stages gives more freedom to the design and development of automatic and adaptive solvers. The adaptation can be bi-directional: on the one hand, with coarse and concise tessellation, the algorithm utilizes the information associated with the original design to obtain more faithful simulations. On the other hand, in early-stage prototyping and automatic data generation for deep learning, the principles established to simplify the shape and retain the settings provide a fast answer for an update. 

For the remainder of this dissertation, Chapter~\ref*{chp:related} reviews related works and our contribution in the broader technical context. 
Following, Chapter~\ref*{chp:scaf} introduces the problem of surface parameterization, where an injective map, from the 3D surface to the 2D domain, assigns a planar u/v coordinate to each point on the surface. The algorithm guarantees the bijectivity between the surface and its codomain, which is essential to uniquely convert surface attached information into a planar representation, and take advantage of widely supported image-based compression and processing techniques could for the color or material. 
The algorithm employs techniques of triangulation augmentation in the virtual expanded domain to fit the bijectivity condition into a mathematical optimization framework, which scales up to large models with ease. 
To demonstrate the practical applicability, we use it to compute globally bijective single patch parametrizations, pack multiple charts into a single UV domain, remove self-intersections from existing models, and deform 3D objects while preventing self-intersections.

Chapter~\ref*{chp:shell} further investigates the problem of bijective attribute transfer between surfaces, where I formulate a novel approach to associate bijective multiple close-by surfaces. The algorithm converts a self-intersection-free, orientable, and manifold triangle mesh into a \emph{generalized prismatic shell} equipped with a bijective projection operator to map the mesh to a class of discrete surfaces contained within the shell whose normals satisfy a simple local condition. 
Properties can be robustly and efficiently transferred between these surfaces using the prismatic layer as a common parametrization domain. 
The combination of the prismatic shell construction and corresponding projection operator is a robust building block readily usable in many downstream applications, including the solution of PDEs, displacement maps synthesis, Boolean operations, tetrahedral meshing, geometric textures, and nested cages.

%

Chapter~\ref*{chp:curve} incorporates the shell into two mesh generation tasks: conforming tetrahedral meshes, and near-surface high-order shell meshes. Combined, they constitute a robust and automatic algorithm to convert dense piece-wise linear triangle meshes with features annotated into coarse tetrahedral meshes with curved elements. The construction guarantees that the high-order meshes are free of element inversion or self-intersection. 
The user can specify a maximal geometrical error from the input mesh, which controls the density of the curved approximation. The boundary of the output mesh is in bijective correspondence to the input, enabling to transfer of attributes between them, such as boundary conditions for simulations, making it an ideal replacement or complement for the original input geometry. We demonstrate the robustness and effectiveness of our algorithm in generating high-order meshes for a large collection of complex 3D models.

While being useful in various applications that I design and investigate, I believe that these geometric constructions can have greater benefit, if incorporated into many existing implementations. While straightforward in theory, such integration often involves error-prone software engineering efforts, especially when our goal is to arrive at robust algorithms. Therefore, an abstraction and novel mesh editing specification paradigm is introduced in Chapter~\ref*{chp:wmtk}. Such an algorithm development toolkit aims to make our contribution accessible to more researchers in the field of robust geometry modeling, and scalable physical simulation.
We introduce a novel approach to describe mesh generation, mesh adaptation, and geometric modeling algorithms relying on changing mesh connectivity using a high-level abstraction. The main motivation is to enable easy customization and development of these algorithms via a declarative specification consisting of a set of per-element invariants, operation scheduling, and attribute transfer for each editing operation.
We demonstrate that widely used algorithms editing surfaces and volumes can be compactly expressed with our abstraction, and their implementation within our framework is simple, automatically parallelizable on shared-memory architectures, and with guaranteed satisfaction of the prescribed invariants. These algorithms are readable and easy to customize for specific use cases.
We introduce a software library implementing this abstraction and providing automatically shared memory parallelization, which utilizes the multi-core structure of the computing resources to accelerate the task of mesh editing.

In addition to the mathematical formulation and algorithmic design, we also provide open-source implementations for the algorithms in the thesis.
Additionally, we release source code for each of the algorithm implementations~\cite{githubScaffoldMap,githubshell,githubbichon}, along with reproducible instructions, to foster further research in this direction.
