Various applications, including artistic creation, scientific computing, and mechanical engineering, require the computing and reasoning of 3D digital objects.
The modeling and processing of 3D geometric shapes, materials, and textures, as well as the simulation of their deformation and interactions within virtual environments, is essential to bring the algorithmic and automated power of computing to real-life manufacture, architecture, and medical device design.
Depending on the specific desiderata and numerical properties, better algorithm designs might prefer 3D data to acquire different representations, for example, in planes, surfaces, or inside volumes.

This thesis investigates the problem related to the representations of data on 3D shapes and across different domains,
so that respective algorithms, especially for different stages within a pipeline, may come together synergistically without manual effort or tuning that disrupts the automation of the data flow.
I propose novel geometrical principles in various geometric modeling and processing stages. 
I also showcase various geometric computing applications that easily integrate such principles to guarantee the geometry validity and algorithm effectiveness of surface parameterization, rendering, deformation/animation, and mechanical simulation.
In addition, by such principles, we can finally explore creative and robust solutions that reliably coarsen the surface. Such informed coarsening accelerates everyday geometric modeling operations; the contribution also includes a stable scheme to construct coarse and curved meshes for fast animation and scientific computing.
Furthermore, the thesis provides a declarative way to formulate mesh processing and adaptation algorithms to facilitate the practical development of robust and reliable mesh processing software.
Besides the algorithm design and the underpinning theory, the thesis includes extensive numerical validations involving tens of thousands of complex geometry shapes. And to maintain replicability and foster further research in this direction, I also released the implementation and generated data to be open source and accessible.
