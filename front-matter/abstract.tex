In various applications ranging from 3D artistic creation, scientific computing, mechanical engineering, and next-generation fabrication pipelines, 
the modeling and processing of 3D geometric shapes, materials, and textures, as well as the simulation of their deformation and interactions within the virtual environment, serve as essential building blocks.
3D data can live in a plane, on a surface, or inside a volume. And each representation can be more desirable for different optimal algorithmic designs and application-specific goals. However, there has been relatively little effort to associate them and further leverage such associations robustly, so algorithms operating on each domain may come together synergistically without manual stabilization of the data flow.
This thesis investigates the problem related to the representations of data on 3D shapes and the data across different domains.

The thesis proposes novel geometrical principles in various geometric modeling and processing stages. Prior algorithms can easily integrate such principles to guarantee the validity and effectiveness of algorithm implementations in surface parameterization, surface deformation, and scientific simulation.
Such guards enhance the robustness and make it easier to build assertions for correctness throughout the error-prone process. It is also helpful for reliably accelerating the bottleneck of computation. 
In addition, the thesis further uses such principles to explore improved solutions for modeling and simulation. 
For example, the thesis describes an algorithm that reliably creates a coarsened surface mesh as a proxy to accelerate everyday geometric modeling operations. The contribution also includes another pipeline that constructs coarse and curved meshes for fast animation and scientific simulations.
Furthermore, the thesis also provides a novel declarative way to formulate mesh processing and adaptation algorithms for the practical adoption of such methodologies. The proposed specification facilitates the development of robust and reliable mesh processing software with a built-in parallelization structure.

Besides the algorithm design, the thesis also explores the underpinning theory to provide guarantees and extensive numerical validations with the implemented software, involving tens of thousands of complex different geometry shapes. Finally, in an effort of reproducibility and fosters further research in this direction, I also released the program implementation and detailed specifications to be open source and accessible.
