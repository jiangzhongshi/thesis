In various applications ranging from 3D artistic creation, scientific computing, mechanical engineering,
the modeling and processing of 3D geometric shapes, materials, and textures, as well as the simulation of their deformation and interactions within virtual environments, serve as essential building blocks.
Better algorithm designs might prefer 3D data to live in planes, on surfaces, or inside volumes, depending on application-specific goals. 
% However, there has been relatively little effort to associate them and further leverage such associations robustly, 
This thesis investigates the problem related to the representations of data on 3D shapes and across different domains,
so that respective algorithms may come together synergistically without manual fixes of the data flow.

I propose novel geometrical principles in various geometric modeling and processing stages. Applications can easily integrate such principles to guarantee the validity and effectiveness of surface parameterization, surface deformation, and scientific simulation.
% Such guards enhance the robustness and make it easier to build assertions for correctness throughout the error-prone process.
In addition, I further leverage such principles to explore efficient solutions in various settings. The thesis introduces an algorithm that reliably coarsens the surface to accelerate everyday geometric modeling operations; the contribution also includes a robust scheme to construct coarse and curved meshes for fast animation and scientific simulations.
Furthermore, the thesis provides a declarative way to formulate mesh processing and adaptation algorithms to facilitate the practical development of robust and reliable mesh processing software.
Besides the algorithm design and the underpinning theory, the thesis includes extensive numerical validations involving tens of thousands of complex geometry shapes. And to maintain replicability and foster further research in this direction, I also released the implementation and genertated data to be open source and accessible.
