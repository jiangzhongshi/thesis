Various applications, from artistic creation, to scientific computing, require the processing and reasoning of 3D digital objects.
The computational modeling of 3D geometric shapes, materials, and textures, as well as the simulation of their deformation and interactions, is essential to bring the algorithmic power of computing to real-life manufacture, architecture, and medical device design.
Depending on the specific numerical properties, better algorithm designs might prefer 3D data with different representations, for example, in planes, surfaces, or inside volumes.
This thesis investigates the problem related to the representations of data on 3D shapes and across different domains,
so computations for different stages within a pipeline may come together synergistically without manual tuning that disrupts an automated data flow.
I propose novel geometrical principles in various geometric modeling and processing stages.
I also showcase various geometric computing applications that easily integrate such principles to guarantee the geometry validity and algorithm effectiveness of surface parameterization, rendering, deformation/animation, and mechanical simulation.
In addition, we can finally explore creative solutions that reliably coarsen the surface. Such simplification accelerates everyday geometric modeling operations; the contribution also includes a scalable method to construct coarse and curved meshes for fast animation and scientific computing.
Furthermore, the thesis provides a declarative way to formulate mesh processing and adaptation algorithms to facilitate the practical development of robust and reliable mesh processing software.
Finally, the thesis includes extensive numerical validations involving tens of thousands of complex geometry shapes. 