
%%% Local Variables:
%%% mode: latex
%%% TeX-master: "thesis"
%%% End:

In various applications from artistic creation, scientific computing, mechanical engineering, and next generation fabrication pipelines, the modeling of 3D geometric shapes, materials, and texture, and the simulation of their deformation and interactions within the virtual environment is essential.
% Of the process, the central representation is the geometry and associated data.
This thesis investigates the problem related to the representations of data on 3D shapes, and the association data across sme different representations. 
Across planar, volume, surface, and curved representations, each of them are more suitable for particular applications. However, the robust ways to associate them, and moreoever leverage the association to has been scarce.  

Such effort makes the algorithms robust and easier to build assertions for correctness throughout the error-prone process. It is also useful for reliably accelearte bottleneck of computation. In addition

The two central technical object in this thesis is the interaction between mesh: discretization of spatial domain, and map: the bijective associations across such domains. I will showcase how they can be closely connected to each other

% There are various approaches to specify a shape, depending on the different geometry characteristics. Shape modeling evolves beyond smooth and regular objects like sculpture or furniture. Algorithm deduced structures have gained popularity since they deliver the promise of maximal durability with limited material. Downstream, computer simulation requires a discrete tessellation of the object. A notable example is a mesh: it contains a list of vertices for the spatial locations and a list of simple polygons or polyhedra. However, it is rarely the case where one mesh fits all. More degrees of freedom give a more accurate description of the desired shape, but it inevitably means more computational resources are needed to handle the computation.
Besides the algorithm design, the thesis also explore the underpinning theory to provide guarantees, along with extensive numerical validations with the implemented software, involving tens of thousands of complex different geometry shapes. Finally, I also release the program implementation and detailed specifications to be open source and accessible.
