In various applications from artistic creation, scientific computing, mechanical engineering, and next-generation fabrication pipelines, 
the modeling of 3D geometric shapes, materials, and textures, as well as the simulation of their deformation and interactions within the virtual environment, serve as important building blocks.
Depending on the application-dependent goal, and optimal algorithmic design, the geometry data may be represented as planar, surface, volume, and even with higher-order representations.
Across planar, volume, surface, and curved representations, each of them are more desirable for different applications. However, there has been relatively little effort to robustly associate them and further leverage such associations.
This thesis investigates the problem related to the representations of data on 3D shapes and the data across some different representations. 

The thesis proposes the use of new geometrical principles in various stages of geometric modeling and processing. The principles can be integrated on top of existing algorithms, to guarantee the validity and effectiveness of algorithmic stages in surface parameterization, surface deformation, and scientific simulation.
Such guards enhance the robustness and make it easier to build assertions for correctness throughout the error-prone process. It is also useful for reliably accelerating the bottleneck of computation. 
In addition, the thesis further makes use of such principles to explore improved solutions for modeling and simulation. 
For example, an algorithm that reliably creates a coarsened surface mesh as a proxy to accelerate the common geometric modeling operations, and another builds coarse and bijective high-order tetrahedral meshes for fast animation and scientific simulations.
Furthermore, for the practical adoption of such methodology, the thesis also provides a novel way to formulate mesh processing and adaptation algorithms, which makes it easier to implement robust and reliable mesh processing software, with a slight overhead cost that can be easily remedied through the built-in parallelization structure.

Besides the algorithm design, the thesis also explores the underpinning theory to provide guarantees, along with extensive numerical validations with the implemented software, involving tens of thousands of complex different geometry shapes. Finally, in an effort of reproducibility, and fosters further research in this direction, I also release the program implementation and detailed specifications to be open source and accessible.
