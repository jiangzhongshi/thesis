In the previous chapters, the dissertation has discussed several problems related to the bijective mappings between different spatial domains and introduced relevant computational principles to guarantee the validity of attribute association and present robust algorithms to process information across digital shapes efficiently. 
In this concluding chapter, I argue that the significance of these principles lies not only during the robust and scalable processing of digital geometry shapes but also in developing the next-generation pipeline of geometric modeling and physical simulation. 
To achieve this broader goal, there are still many open and exciting problems to solve.

\begin{enumerate}
  \item \emph{Shape-Aware Adaptive Mesh Refinement}:
In chapter~\ref{chp:curve}, we showcased the benefit of surface and volume simplification in the context of scalable visualization and efficient physical simulation. 
On the other hand, we can further extend the pipeline to refine back towards a given geometry by adding detail and smoothness adaptively. Combined with an Adaptive Mesh Refinement (AMR) pipeline, such features are essential for scientific computing and computational design.
Specifically, given a detailed reference shape with material assignment, the algorithm in chapters~\ref{chp:shell},\ref{chp:curve} coarsen the volume domain with drastically fewer elements. 
The governing equations with transferred boundary conditions may be efficiently solved in such a simplified domain. The geometry might refine where needed depending on the current solution's smoothness or accuracy. 
Note that such refinement should always converge to the reference shape to bring the geometry error down, and finally, we can achieve an adaptive solution on an otherwise dense geometry. 

\item \emph{Bijective Shell in the Wild} % partial shell parsely
In chapter~\ref{chp:shell}, the bijective shell is formulated on manifold meshes, 
to guarantee strictly bijective attribute transfer everywhere.
However, a blended framework with envelope \cite{hu2018tetrahedral, Wang:2021} can open the door to partially corrupted or occluded data. 
Such characteristics are common in 3d scanned datasets or MRI medical datasets, and point cloud and implicit represented surface data may all be processed within the same framework.
Given an input mixed shape description, the algorithm would first segment the partial regions with attributes and apply the bijective shell construction algorithm on the partial surface. During downstream remeshing, different parts would be checked with varying envelope criteria, and extra care should be taken to ensure continuity across the segmentation boundaries. 

\item \emph{High-Order reduced FEM simulations}
The high order finite element method is a concept already with decades of development. Still, we believe that the availability of more high-order surface and volume meshes (chapter~\ref{chp:curve}) would benefit its adoption and release the potential inefficient physical computations. In addition, resolving frictional contact with curved representations, \cite{ferguson2022high} can make dynamic simulations more scalable and efficient. 
While surface modeling software often uses curved geometries (e.g. B{\'e}zier and NURBS), the conversion of these geometries to curved simulation meshes is still an open problem, and existing solution are not robust. In this aspect, our contribution of robust and intersection-free curve surface modeling can lead to a more systematically robust industrial design and simulation pipeline.

\item \emph{Improved Computational Fabrication of Customized Devices}
The chapters~\ref{chp:shell},\ref{chp:curve} have introduced the robust association between design and simulation representations. A closely related concept with attribute transfer is differentiability. Novel schemes for differentiable physics on the same mesh has been proposed. And in the presence of large deformation with connectivity change, our construction comes in handy by bijectively associating through the remeshing process. 
With such end-to-end associations, the initial design decisions in the shape space can be reflected in the final durability of the parts, and differentiability ensures that the design can be instructed to improve automatically.  

\item \emph{Volume-Aware 3D Geometry Processing}
can be another venue for interesting future works. 
Similar to the principle we explained in chapter~\ref{chp:scaf}, a background mesh provides implicit and easy ways to check for the geometry validity. For example, the scaffold mesh converts global surface self-intersection into locally inversion checks. It has the potential to unify the framework in different shapes and orders of finite element cells, and is partially adopted in the construction of curved shell~\ref{chp:curve}. Traditional geometry processing algorithms, including smoothing, fairing, and animation, can be more efficient and reliable under topology constraints with a tetrahedral-based virtual domain. A notable example is the surface insertion or mesh arrangement problem we used in Chapter~\ref{chp:wmtk}: the exact version is still the bottleneck of our tetrahedral mesh generation algorithm, but recent advances in mesh arrangement~\cite{Hu:2019:fTetWild,zhou2016mesh,ember2022} are promising on this path.

\item \emph{Robust Large-Scale Geometry Dataset Processing}
Finally, advances in new processing algorithms are necessary for improved and automatic data collection techniques. 
Better data is the stepping stone to fully leveraging information from the large-scale dataset~\cite{koch2019abc,zhou2016thingi10k}. 
More significant amounts of data pave the way for more tasks, including semantic shape segmentation, point cloud feature detection, object physics with interaction understanding, and robotics manipulation.

\end{enumerate}

