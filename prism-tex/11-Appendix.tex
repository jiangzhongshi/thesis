
\section{Bijectivity of the Nonlinear Prismatic Transformation}
\label{app:bilinear}
In this section, we show
that the isoparametric transformation of prismatic finite element \cite{ciarlet1991basic} is also bijective if I1 holds.%(on each half of the shell).

We follow the notation from {Appendix~\ref{app:I1}.} 
The map is defined as (for the top slab)
\begin{align*}
    f(u,v,\eta) &= u e_1 + v e_2  + \eta n_0 + u \eta {n_{01}} +
    v \eta {n_{02}},\\
    \det J_f &=  \langle (1-u-v) n_0 + u n_1 + v n_2, e_1 + \eta {n_{01}},
    e_2 + \eta {n_{02} \rangle,}
\end{align*}
where $\eta$ is the variable corresponding to the thickness direction{, and $n_{ij} = n_j - n_i$.}
Observe that $\det J_f(u,v,\eta_0)$ is linear in $u,v$ for fixed $\eta_0$, the extrema will only be achieved at the corner points \cite{knabner2001invertibility}. Therefore, it is sufficient to check the signs at three edges
$(0,0,\eta)$, $(1,0,\eta)$, $(0,1,\eta)$ for $\eta \in [0,1]$.
\begin{align*}
    & \det J_f(0,0,\eta) \\
    &= \langle n_0, (1-\eta)e_1 +\eta (e_1 + n_1 - n_0),(1-\eta)e_2 +\eta (e_2 + n_2 - n_0) \rangle \\
    &= (1-\eta)^2\langle n_0, e_1, e_2 \rangle + 
    (1-\eta)\eta \langle n_0,e_1, e_2+n_2\rangle \\&+ 
    (1-\eta)\eta \langle n_0, e_1+n_1, e_2\rangle + 
    \eta^2\langle n_0, e_1+n_1, e_2 + n_2\rangle\\
    &= (1-\eta)^2 \text{vol}_4 + (1-\eta)\eta (\text{vol}_3 +\text{vol}_1) + \eta^2 \text{vol}_2 > 0.
\end{align*}

By symmetry, 
it is easy to verify {that the following are positive},
\begin{align*}
    \det& J_f(1,0,\eta) = (1-\eta)^2 \text{vol}_6 + (1-\eta)\eta (\text{vol}_7 +\text{vol}_5) + \eta^2 \text{vol}_8{,} \\
    \det& J_f(0,1,\eta) = (1-\eta)^2 \text{vol}_{11} + (1-\eta)\eta (\text{vol}_9 +\text{vol}_{10}) + \eta^2 \text{vol}_{12}.\\
\end{align*}


\section{Double Slab}\label{app:dblayer}
In Section~\ref{prism:sec:projection}, we explain that we want to ensure that our shell validity is independent from the enumeration of the faces; for this reason, we require that the conditions I1 and I2 are satisfied for all possible decompositions. Without using the double slab, different decompositions of the prism will result in  different intersections with the middle surface. In other words, the projection operator will project the input mesh to a different set of triangles on the middle surface. 
The double slab makes the middle surface independent of the decomposition by construction. 
We note that Theorem~\ref{thm:bevel} is valid only for double-slab prisms since the statement requires that one prism contains only one triangle.

\section{Variants}
(Disclaimer: this section is unpublished material. And serves as post publish notes for future references.)
\subsection{Pointless Variant}
We extend the framework, by allowing the relaxation of definition of a section. 
Originally, any intersection between the triangle and the (different decomposition of the) prism is counted. Here, we explicitly forgive when the intersection is only one point on the vertex of mid surface and the triangle. 
This predicate can be easily implemented, making use of the bitwise equal coordinates.
Therefore, the initial tracking section only follows edge-adjacency relationship, instead of the vertex adjacent one in the main text. 
With such adjusted condition for section, the optimization can be trivially accomodated. The only thing that is different (simpler) is the part for bevel, since we have less triangles to consider.
The existing one is still valid, however, we can get away with simpler red-green intersection

\subsection{Dynamic Intersection Check}
To make easier the guarantee of a self-intersection-free shell, substitue the AABB tree collision check with a dynamic hashgrid. We initialize two hashgrids at the beginning, one for top surface and another one for bottom surface. This is followed by additional shrinking to make sure they are clear. Each local opeartion is responsible to trial and keep the validity of the surfaces. Furthermore, the top/bottom surface should not interfere with the reference input surface, this can be accelerated through the maintainance of the tracking list, and is trial checked before a local operation is performed.

\subsection{Positive Detereminant for Natural Prism Map}
This section follows and extends Appendix~\ref{app:bilinear}
In fact, we observe that, for a quadratic bezier curve piece $a t^2 + b t (1-t) + c (1-t)^2, t\in [0,1]$, the sufficient and necessary condition is $a>0, c>0, b > - 2 \sqrt{a c}$, and the last relation is equivalent to (but more clearly stated as) $b > 0 || b^2 - 4 a c < 0$.
The current text (all 12 positive tetra) impose an unnatural constraint on edge splits: in the case of an agressively positive prism (all decompositions are all-positive), sub-prism may not be positive in the same sense. While this will not break anything, forbidding edge split will lead to unexpected lower quality in the optimization procedure.
