% ----------------------------------------
% Packages
% ----------------------------------------

% 
% Place here your \usepackage's. Some recommended packages are already included.
%

% Graphics:
\usepackage[final]{graphicx}
%\usepackage{graphicx} % use this line instead of the above to suppress graphics in draft copies
%\usepackage{graphpap} % \defines the \graphpaper command

% Uncomment this to indent first line of each section:
% \usepackage{indentfirst}

% Good AMS stuff:
\usepackage{amsthm} % facilities for theorem-like environments
\usepackage[tbtags]{amsmath} % a lot of good stuff!

% Fonts and symbols:
\usepackage{amsfonts}
\usepackage{amssymb}

% Set the fonts
\RequirePackage[T1]{fontenc}
\ifxetex
  \RequirePackage[tt=false]{libertine}
\else
  \RequirePackage[tt=false, type1=true]{libertine}
\fi
\RequirePackage[varqu]{zi4}
\RequirePackage[libertine]{newtxmath}


% For typesetting inference rules
\usepackage{mathpartir}
% \usepackage{pftools}  % A local package
\newcommand{\bmmax}{2}
\usepackage{bm}

% Formatting tools:
%\usepackage{relsize} % relative font size selection, provides commands \textsmalle, \textlarger
%\usepackage{xspace} % gentle spacing in macros, such as \newcommand{\acims}{\textsc{acim}s\xspace}

% Page formatting utility:
%\usepackage{geometry}

\usepackage{booktabs}   %% For formal tables:
                        %% http://ctan.org/pkg/booktabs
\usepackage[labelformat=simple]{subcaption} %% For complex figures with subfigures/subcaptions
                        %% http://ctan.org/pkg/subcaption
% Options to subcaption are to label and refer to subfigures as Fig 1(a) etc.
\renewcommand\thesubfigure{(\alph{subfigure})}

\usepackage[T1]{fontenc} % needed for scaling fancy fonts (?)
\usepackage[utf8]{inputenc} % not sure this is needed

\usepackage{amssymb}
%\usepackage[table]{xcolor}

% For code
\usepackage[final]{listings}
\lstset{mathescape=true}

% For code highlighting
% \usepackage{bold-extra}

% Tikz
\usepackage{tikz}
\usetikzlibrary{matrix,arrows,positioning,calc,fit,backgrounds}

% To control enum item labelling/numbering
\usepackage[shortlabels, inline]{enumitem}
% To give custom item labels and reference them
\makeatletter
\newcommand{\myitem}[1][]{
  \protected@edef\@currentlabel{#1}%
\item[#1]
}
\makeatother

% To stop aligned env swallowing up []s
\usepackage{mathtools}

% To use ifstrempty
\usepackage{etoolbox}

% For math mode tables
\usepackage{array}
% A text column in array
\newcolumntype{L}{>$l<$}

% For \llbracket and \rrbracket
\usepackage{stmaryrd}

% For dashed boxes
\usepackage{dashbox}

% For big separating conjunction
\usepackage{scalerel}

% For mathpar environment
\usepackage{mathpartir}

\usepackage{xspace}
\usepackage{multirow}

% To stop citations overflowing lines
\usepackage{breakcites}

% For citet command
\usepackage{natbib}
\setcitestyle{%
    authoryear,%
    open={[},close={]},citesep={;},%
    aysep={},yysep={,},%
    notesep={, }}
\let\cite\citep

%%
%% Place here your \newtheorem's:
%%

\theoremstyle{plain}
\newtheorem{theorem}{Theorem}[chapter]
\newtheorem{conjecture}[theorem]{Conjecture}
\newtheorem{proposition}[theorem]{Proposition}
\newtheorem{lemma}[theorem]{Lemma}
\newtheorem{corollary}[theorem]{Corollary}
\theoremstyle{definition}
\newtheorem{example}[theorem]{Example}
\newtheorem{definition}[theorem]{Definition}
\theoremstyle{plain}


% ----------------------------------------
% Generic definitions
% ----------------------------------------
% Required packages: listings, tikz

% A footnote without a marker
\newcommand\blfootnote[1]{%
  \begingroup
  \renewcommand\thefootnote{}\footnote{#1}%
  \addtocounter{footnote}{-1}%
  \endgroup
}

\renewcommand{\le}{\leqslant}
\renewcommand{\ge}{\geqslant}
% \renewcommand{\emptyset}{\ensuremath{\varnothing}}
% \newcommand{\ds}{\displaystyle}

% ----------------------------------------
% Paper specific macros & commands
% ----------------------------------------


% Put your definitions here


%%% Local Variables:
%%% mode: latex
%%% TeX-master: "thesis"
%%% End:

\usepackage{color}
\usepackage{amssymb}
\usepackage{amsmath}
\usepackage{url}
\usepackage[utf8]{inputenc}
\usepackage{appendix}
\usepackage{pifont}
\usepackage{algorithm2e}
\usepackage[normalem]{ulem}


\renewcommand{\topfraction}{0.9}     % max fraction of floats at top
\renewcommand{\bottomfraction}{0.8}     % max fraction of floats at bottom
\setcounter{topnumber}{4}
\setcounter{bottomnumber}{4}
\setcounter{totalnumber}{8}     % 2 may work better
\setcounter{dbltopnumber}{4}    % for 2-column pages
\renewcommand{\dbltopfraction}{0.95}     % fit big float above 2-col. text
\renewcommand{\textfraction}{0.07}     % allow minimal text w. figs
\renewcommand{\floatpagefraction}{0.7}     % require fuller float pages
\renewcommand{\dblfloatpagefraction}{0.7}     % require fuller float pages

\usepackage{microtype}
\usepackage{wrapfig}

\newcommand{\remove}[1]{} 
\newcommand{\revision}[1]{#1} 

%%% Math Symbols
\newcommand{\T}{\mathcal{T}}
\renewcommand{\P}{\mathcal{P}}
\renewcommand{\S}{\mathcal{S}}
\newcommand{\M}{\mathcal{M}}
\newcommand{\V}{\mathcal{V}}
\newcommand{\Tet}{\boldsymbol{T}}
\renewcommand{\H}{\mathcal{H}}
\newcommand{\C}{\mathcal{C}}
\newcommand{\N}{\mathcal{N}}
\newcommand{\Prism}{\Delta}
\newcommand{\D}{\mathcal{D}}
\newcommand{\J}{\boldsymbol{J}}

% \newtheorem{proposition}[theorem]{Proposition}

% \DeclareMathOperator*{\argmin}{arg\,min}
\DeclareMathOperator*{\tr}{tr}
\DeclareMathOperator*{\isinside}{is\_inside}
\DeclareMathOperator*{\issection}{is\_section}

\usepackage{graphicx}

\graphicspath{{prism-tex/figs/}}
