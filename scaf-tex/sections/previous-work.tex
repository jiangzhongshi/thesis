\section{Previous Work}
\label{sec:related}
Bijective maps find a host of applications in a variety of fields including physical simulation, surface deformation, and parametrization.  We review only the most relevant prior works here and refer to the following surveys for more details~\cite{FloaterSurvey:2005,Sheffer:2006,Hormann:2007}.

\subsection*{Locally Injective Maps.}
 There are many methods that focus on creating \revision{locally} injective maps, which amounts to requiring that triangles maintain their orientation (i.e. they do not flip). In mesh parameterization, many flip-preventing metrics have been developed: the idea is to force the metric to diverge to infinity as triangles become degenerate, inhibiting flips.  These metrics optimize various geometric properties such as angle~\cite{Hormann:2000,Degener:2003} or length~\cite{Sander:2001,Sorkine:2002,Aigerman:2014,Poranne:2014,Smith:2015} preservation.  Similar techniques in the context of deformation have been used to add barrier functions to enforce local injectivity in deformations~\cite{Schuller:2013}. Our method uses these techniques to prevent flips in the scaffold.

Many methods have also been developed to optimize these distortion energies including moving one vertex at a time~\cite{Hormann:2000}, parallel gradient descent~\cite{Fu:2015}, as well as other quasi-newton approaches~\cite{Smith:2015,Kovalsky:2016,Rabinovich:2017}.  Other approaches construct such maps by performing a change of basis, projecting to an inversion free space, and then constructing a parametrization from the result~\cite{Fu:2016}. While our method could potentially use any of these optimization methods, we use \cite{Rabinovich:2017} for its large step sizes. We elaborate on this choice in Section~\ref{sec:general_form}.

\subsection*{Bijective Maps.}
In addition to injective constraints, bijective maps have the additional requirement that the boundary does not intersect.  One simple method for creating a bijective map in 2D involves constraining the boundary to a convex shape such as a circle~\cite{Tutte:1963,Floater:97}.  Such parametrizations guarantee a bijective map in 2D but create significant distortion.  Even so, these methods are commonly used to create a valid starting point for further optimization \cite{Schuller:2013,Smith:2015,Rabinovich:2017}.  While methods that produce bijective maps with fixed boundaries exist~\cite{Weber:2014,Campen:2016}, we aim to produce maps where the boundary is free to move to reduce the distortion of the map.  

\revision{\cite{Gotsman:2001, surazhsky2001morphing} introduced the concept of scaffolding where the free space is triangulated for the purpose of morphing without self-intersection.} \revision{In \cite{Zhang:2005}, t}he scaffold triangles are given a step function for their error: zero \revision{if not flipped, otherwise} infinity. Hence, the bijective condition becomes local in that the shape can evolve until a scaffold triangle flips, in which case the free space is retriangulated and the optimization continues. The main limitation of this work is the lack of an evolving triangulation during the line search and the absence of a rotationally invariant metric for the scaffold triangles, which lead to very small steps and an inefficient optimization.

The Deformable Simplicial Complex (DSC) method~\cite{Misztal:2012} utilize a triangulation of both the free space and the interior of an object to track the interface between the two volumes.  Similar to \cite{Zhang:2005}, the DSC retriangulates at degeneracies but also performs operations to improve the shape of the triangles.  This method changes the triangulation of the interface that it tracks, which works well for simulation, but it is not allowed in many other applications such as UV mapping.

Air meshes~\cite{Muller:2015} extends the technique of Zhang et al.~\cite{Zhang:2005} to add the concept of triangle flipping based on a quality measure during the optimization instead of simply retriangulating at the first sign of a degeneracy. However, this method does not maintain bijective maps as boundaries are allowed to inter-penetrate during optimization: the scaffold is only used to efficiently detect problematic regions, and the local injectivity requirement is a soft constraint in the optimization. The problem tackled in this chapter is much harder, because we do not allow any overlap during any stage of the optimization to guarantee that the resulting maps will be bijective.

\citep{Smith:2015} take a different approach: instead of using a scaffold triangulation, the authors introduce a locally supported barrier function for the boundary to prevent intersection and explicitly limit the line search by computing the singularities of both the distortion energy and the boundary barrier function. Such an approach is inspired by traditional collision detection and response methods that are discussed below. Given a  bijective starting point, this approach never leaves the space of bijective maps during optimization.  Its main limitation is that it is computationally expensive, especially for large models. Our method is two order of magnitude faster (Figure \ref{scaf:fig:smith}).

\subsection*{Collision Detection and Response.}
While not directly related to our approach, bijective maps inherently  involve some form of collision detection and response to avoid overlaps.  The field on collision detection is vast, and we refer the reader to a survey~\cite{jimenez:2001}. In terms of simulations, methods such as asynchronous contact mechanics~\cite{Harmon:2009,harmon2010robust,Ainsley:2012} ensure the bijective property but are very expensive and designed to operate as part of a simulation. Differently, our approach is specialized for geometric optimization, where we are interested in a  quasi-static solution (i.e. we do not want to explicitly simulate a dynamic system, but only find an equilibrium solution).

The work that is closer to ours in term of application (but very different in term of formulation) is \cite{Harmon:2011}, where collision detection and response is used to interactively deform shapes while avoiding self-intersections. Similarly to the previous methods, the explicit detection and iterative response is expensive when many collisions happen at the same time.  Our work avoids these expensive computations, and can robustly handle hundreds of simultaneous collisions while still making large steps in the optimization.

\subsection*{Seam Creation.}
In the context of parametrization, some approaches optimize the connectivity of the charts of the surface during parametrization to obtain a bijective map.  \cite{Levy:2002,Zhou:2004} parameterize the surface and then split charts based on whether they intersect~\cite{Levy:2002} or based on a level of distortion~\cite{Zhou:2004}.  Sorkine et al.~\cite{Sorkine:2002} employ a bottom-up approach and add triangles to a parametrization chart until bijectivity would be violated. The problem we are solving is more general (seams are only useful for texture mapping applications) and constrained (we preserve the prescribed seams). Our algorithm could be used by these algorithms to parametrize single charts, which could reduce the number of additional seams.

